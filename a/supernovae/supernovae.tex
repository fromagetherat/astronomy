\section{Supernovae}
\begin{itemize}[noitemsep]
	\item massive stars or white dwarfs are destroyed as part of an explosion occurring during their last evolutionary stages of life
	\item material is expelled from the star at up to 300,000 km/s, 10\% the speed of light
	\item the explosion creates a shock wave which sweeps up an expanding shell of gas and dust, creating a \textbf{supernova remnant}
\end{itemize}


\subsection{Interstellar Impact}
\subsubsection{Heavy Elements (Supernova Nucleosynthesis)}
\begin{itemize}[noitemsep]
	\item supernovae are the major source of elements heavier than nitrogen
		\begin{enumerate}
			\item \textbf{nuclei up to $^{34}$S} are produced by nuclear fusion
			\item \textbf{nuclei between $^{36}$Ar and $^{56}$Ni} are produced by silicon photodisintegration rearrangement and quasiequilibrium during silicon burning
			\item \textbf{nuclei of elements heavier than iron} are produced by the rapid capture of neutrons during the supernova's collapse
		\end{enumerate}
	\item most likely sites of the \textbf{r-process} --- rapid capture of neutrons that occurs at high temperature and high density of neutrons, which produces unstable nuclei rich in neutrons which beta decay into more stable forms
		\begin{itemize}[noitemsep]
			\item accounts for half of all heavier isotopes of elements beyond iron
		\end{itemize}
\end{itemize}

\subsubsection{Stellar Evolution}
\begin{itemize}[noitemsep]
	\item shockwaves created by the explosion sweep up surrounding interstellar medium during a 2 century phase
	\item the wave undergoes \gls{adiabatic expansion}*, mixing w/ the interstellar medium for 10,000 years
	\item metals created by the supernovae enrich the molecular clouds where stars form $\rightarrow$ each successive generation of stellar formation has a more metal-rich composition
	\item kinetic energy from a supernova remnant may also trigger star formation  by compressing dense molecular clouds in space
\end{itemize}

\subsection{Supernova Taxonomy}
supernovae are classified according to their light curves and absorption lines of the chemical elements appearing in their spectra. if a supernova's spectrum contains lines of hydrogen (the \textbf{Balmer series}), it is classified as Type II; otherwise, it is Type I.
\subsubsection{Type Ia Supernovae}
Type Ia supernovae present a singly ionized silicon (Si II) line at 615.0 nm (nanometers), near peak light. their progenitor stars are considered to be white dwarfs.

2 models for their formation exist:
\begin{enumerate}
	\item the \textbf{single degenerate progenitor} model postulates that Type Ia supernovae form from close binary star systems and account for 20\% of Type Ia supernovae
		\begin{enumerate}
			\item the larger, primary star evolves into a giant, expanding its envelope
			\item the primary star spills gas onto the secondary star, engulfing it
			\item during this shared-envelope phase, the secondary star and the core of the primary spiral toward each other
			\item the gas forming the common envelope is ejected
			\item the core of the primary star collapses and becomes a white dwarf
			\item the companion star, now larger than the remaining core, swells and spills gas onto the white dwarf
			\item the white dwarf's mass increases until it reaches the \gls{Chandrasekhar limit}*, at which point the white dwarf's electron degeneracy pressure is unable to prevent collapse, and it explodes, ejecting the companion star
		\end{enumerate}
	\item the \textbf{double degenerate progenitors} postulates that Type Ia supernovae form from the merger of 2 white dwarfs whose combined mass exceeds the \gls{Chandrasekhar limit}*.
		\begin{enumerate}
			\item collisions occur b/w a binary star system or 2 binary systems containing white dwarfs.
			\item the collision creates a binary system of 2 white dwarfs.
			\item the orbit of the 2 stars decays and they merge through their shared envelopes.
			\item the combined white dwarf's electron degeneracy pressure is unable to prevent collapse, and it explodes.
			\item studies find that white dwarf mergers occur every 100 years in the Milky Way, matching the number of Type Ia supernovae.
			\item observations from NASA telescopes rule out the possibility of existing supergiant or giant companion stars in the supernovae studied, whose outer shells would emit X-rays had they existed.
		\end{enumerate} 
\end{enumerate}
regarding the mechanics of the explosion:
\begin{itemize}[noitemsep]
	\item due to the accretion of material onto the white dwarf, increased pressure and density raise the temperature of the core
	\item when the star approaches 99\% of the \gls{Chandrasekhar limit}*, a 1,000 year period of convection begins
	\item a deflagration (layered burning) flame ignites, leading to carbon and oxygen fusion
	\item since white dwarfs depend on degeneracy pressure, which is independent of temperature, they are vulnerable to runaway fusion reactions $\rightarrow$ the carbon and oxygen quickly fuses into heavier elements, leading to a massive energy output which unbinds the star
\end{itemize}
Type Ia supernovae tend to appear in all types of galaxies, whether or not stellar formation is ongoing. this is because a system which has lived long enough to not only form white dwarfs but also transfer enough mass to explode is likely to migrate far from the regions where it formed.

Type Ia supernovae have a characteristic \gls{light curve}*. near maximal luminosity, the spectrum contains lines of intermediate-mass elements from oxygen to calcium, since these comprise the outer layers of the star. some time afterward, when the outer layers have expanded enough to become transparent, the spectrum consists of heavy elements synthesized from materials near the core of the star. the peak is primarily powered by the decay of nickel, while the later stage is powered by cobalt.
\begin{center}
\includegraphics[scale=1]{supernovae/"Ia curve"}
\end{center}

\subsubsection{Type II Supernovae}
type II supernovae are distinguished from other supernovae by the presence of hydrogen in their spectra. their progenitor stars are much larger, at least 8 times but no more than 40-50 times, larger than the sun of the Milky Way.

the energy to create them is generated through nuclear fusion:
\begin{enumerate}
	\item stars with great enough mass can fuse elements w/ atomic mass higher than hydrogen and helium (as opposed to our Sun).
	\item the energy of the fusion reaction combined with electron degeneracy pressure maintains stellar equilibrium for a time.
	\item eventually, fusion produces a core of iron and nickel.
	\item fusion of these 2 elements produces no energy output, so the core becomes inert.
	\item w/o energy to create thermal pressure, gravity causes the core to contract until it exceeds the \gls{Chandrasekhar limit}*, leading to implosion of the inner core.
	\item lacking the support of the collapsed inner core, gravity causes the outer core to collapse, increasing the temperature up to 100 billion kelvin. as the core's density increases, it becomes energetically favorable for electrosn and protons to merge.
	\item neutrons and neutrinos are formed through beta-decay. neutrinos, which rarely interact w/ normal matter, escape from the core, carrying away energy which accelerates the collapse. some neutrino's are absorbed by the star's outer layer, beginning the explosion.
	\item the resulting neutron degeneracy (neutron-neutron repulsive interactions) stops the collapse of the inner core, causing the implosion to rebound outward.
	\item the shockwave output unbinds the star, leading to an explosion.
	\item for a brief time, the high temperature and pressure allows for nucleosynthesis of elements heavier than iron.
	\item depending on initial size of the star, the remnants of the core form a neutron star or a black hole.
\end{enumerate}
type II supernovae are usually observed in the spiral arms of galaxies and in H II regions, but not in elliptical galaxies.

the \gls{light curve}* of type II supernovae demonstrates a characteristic rise to peak luminosity followed by decline. the average decay rate of 0.008 magnitudes per day is much lower than the decay rate of type I supernovae. they can be separated into 2 categories. 
\begin{enumerate}
	\item the \gls{light curve}* of a \textbf{type II-L} supernova shows a linear decline following peak luminosity. the net decay rate is 0.012 magnitudes/day. this is b/c most of the hydrogen envelope in the progenitor star has been expelled.
	\item the \gls{light curve}* of a \textbf{Type II-P} supernova shows a distinctive plateau during the decline where luminosity decreases at a slower rate. the net decay rate is 0.0075 magnitudes/day. this is because the progenitor star maintains its hydrogen envelope. when the shockwave oxidizes this hydrogen, it increases the opacity, preventing photons from the inner explosion from escaping. later, when it cools, the hydrogen recombines and becomes transparent once more.
\end{enumerate}
\begin{center}
\includegraphics[scale=1]{supernovae/"II curve"}
\end{center}