\subsubsection{Elliptical Galaxies}
\begin{longtabu} to \textwidth{| | p{3.5cm} | X | |}
\hline
	General Characteristics &
	\begin{itemize}[noitemsep]
		\item smooth, featureless image comprised of ovoid masses of stars attached by the gravitational attraction b/w them
		\item no rotational axis --- stars show wide range of orbital paths around center, primarily radial motion; slight uniformity is what determines overall shape of the galaxy
	\end{itemize}
	\\
	\hline
	Stars &
	\begin{itemize}[noitemsep]
		\item ellipticals contain mostly \textbf{old stars}
			\begin{itemize}[noitemsep]
				\item more red in color
				\item very little gas and dust hampers formation of new stars
			\end{itemize}
	\end{itemize}
	\\
	\hline
	Shapes and Sizes &
	\begin{itemize}[noitemsep]
		\item highest variability of all galaxy types:
			\begin{itemize}[noitemsep]
				\item wide range of masses --- $10^{5}$ to $10^{13}$ solar masses
				\item wide range of sizes --- observations showing that objects can have diameters of between 1 and 100 kiloparsecs (or 3260 to 326,000 light years)
				\item wide range of brightnesses --- some can be up to 10 times brighter than the brightest spirals. At the other end of the scale, the faintest ellipticals can be 1000 times less luminous than the faintest spirals
			\end{itemize}
		\item The Hubble classification of elliptical galaxies contains an integer, $n$ that describes how elongated the galaxy image is. The classification is determined by the ratio of the major (a) to the minor (b) axes of the galaxy's \gls{isophote}s: $ 10 \times (1 - \frac{b}{a}) $
		\item thus, a given elliptical galaxy can be classified as $E_{n}$, where an $E_0$ galaxy is spherical, and an $E_7$ galaxy is flat. this classification is dependent on the angle from which the galaxy is viewed and thus does not affect its physical properties, but is useful for describing how a galaxy appears through a telescope.
	\end{itemize}
	\\
	\hline
	Evolution &
	\begin{itemize}[noitemsep]
		\item astronomers believe that elliptical galaxies form earlier than spiral galaxies, but they can still have quantities of gas and dust, and can still be very noisy in the radio spectrum. evidence has shown that a reasonable proportion (~25\%) of early-type (E, ES and S0) galaxies have residual gas reservoirs and low level star-formation.
		\item evolve from the fusion of smaller, gravitationally bound galaxies which are of similar size
		\item more commonly found around clusters and groups of galaxies due to forming from fusion. They are less frequently spotted in the early universe, which supports the idea that they evolved from the collisions that came later in the life of a galaxy.
		\item A supermassive black hole is thought to lie at the center of these ancient galaxies. These gluttonous giants consume gas and dust, and may play a role in the slower growth of elliptical galaxies.
	\end{itemize}
	\\
	\hline
\end{longtabu}

\subsubsection{Dwarf Elliptical Galaxy}
\begin{longtabu} to \textwidth{| | p{3.5 cm} | X | | }
\hline
General Characteristics &
 elliptical galaxies that are smaller than ordinary elliptical galaxies. They are quite common in galaxy groups and clusters, and are usually companions to other galaxies. Low-luminosity elliptical galaxies are distinguished from late-type galaxies (spirals and irregulars) by their smooth surface-brightness profiles.
Despite their name, dwarf ellipticals are not really fainter versions of true elliptical galaxies, but are structurally distinct.
 \\
 \hline
 Shapes and Sizes &
 Dwarf elliptical galaxies have blue absolute magnitudes within the range ?18 mag < M < ?14 mag, fainter than ordinary elliptical galaxies. Below luminosities of MB approx -18 the smooth-profile galaxies divide into two classes: compact galaxies with high central surface brightnesses (exemplified by M32), and diffuse galaxies with low central surface brightnesses (exemplified by the Local Group dwarf spheroidals).
 
 Typical dE's have masses of about one billion solar masses, or about 1/1000th that of a typical giant galaxy. They contain very little or no gas, which makes them different from dwarf irregular galaxies. Three relatively bright dE's are known in the Local Group: NGC 147, 185, and NGC 205, all companions of the Andromeda Galaxy. Hundreds of similar galaxies exist in the relatively nearby Virgo Cluster.
 \\
 \hline
 Evolution &
 \begin{itemize}[noitemsep]
 	\item thought to be primordial objects built from the coalescing of dark matter and gas objects to form the building blocks of ordinary galaxies
	\item alternately, they could be the remains of low-mass spiral galaxies that were transfigured into a rounder shape through repeated \gls{galaxy harassment}* from ordinary galaxies within a cluster. vidence for the hypothesis had been claimed by studying early-type dwarf galaxies in the Virgo Cluster and finding structures, such as disks and spiral arms, which suggest they are former disk systems transformed by the above-mentioned interactions. However, the existence of similar structures in isolated early-type dwarf galaxies, such as LEDA 2108986, has undermined this hypothesis.
\end{itemize}
\\
\hline
\end{longtabu}