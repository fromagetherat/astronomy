\subsubsection{Elliptical Galaxies}
\begin{longtabu} to \textwidth{| | p{3.5cm} | X | |}
\hline
	General Characteristics &
	\begin{itemize}[noitemsep]
		\item smooth, featureless image comprised of ovoid masses of stars attached by the gravitational attraction b/w them
		\item no rotational axis --- stars show wide range of orbital paths around center, primarily radial motion; slight uniformity is what determines overall shape of the galaxy
	\end{itemize}
	\\
	\hline
	Stars &
	\begin{itemize}[noitemsep]
		\item ellipticals contain mostly \textbf{old stars}
			\begin{itemize}[noitemsep]
				\item more red in color
				\item very little gas and dust hampers formation of new stars
			\end{itemize}
	\end{itemize}
	\\
	\hline
	Shapes and Sizes &
	\begin{itemize}[noitemsep]
		\item highest variability of all galaxy types:
			\begin{itemize}[noitemsep]
				\item wide range of masses --- $10^{5}$ to $10^{13}$ solar masses
				\item wide range of sizes --- observations showing that objects can have diameters of between 1 and 100 kiloparsecs (or 3260 to 326,000 light years)
				\item wide range of brightnesses --- some can be up to 10 times brighter than the brightest spirals. At the other end of the scale, the faintest ellipticals can be 1000 times less luminous than the faintest spirals
			\end{itemize}
		\item The Hubble classification of elliptical galaxies contains an integer, $n$ that describes how elongated the galaxy image is. The classification is determined by the ratio of the major (a) to the minor (b) axes of the galaxy's \gls{isophote}s: $ 10 \times (1 - \frac{b}{a}) $
		\item thus, a given elliptical galaxy can be classified as $E_{n}$, where an $E_0$ galaxy is spherical, and an $E_7$ galaxy is flat. this classification is dependent on the angle from which the galaxy is viewed and thus does not affect its physical properties, but is useful for describing how a galaxy appears through a telescope.
	\end{itemize}
	\\
	\hline
	Evolution &
	\begin{itemize}[noitemsep]
		\item astronomers believe that elliptical galaxies form earlier than spiral galaxies, but they can still have quantities of gas and dust, and can still be very noisy in the radio spectrum. evidence has shown that a reasonable proportion (~25\%) of early-type (E, ES and S0) galaxies have residual gas reservoirs and low level star-formation.
		\item evolve from the fusion of smaller, gravitationally bound galaxies which are of similar size
		\item more commonly found around clusters and groups of galaxies due to forming from fusion. They are less frequently spotted in the early universe, which supports the idea that they evolved from the collisions that came later in the life of a galaxy.
		\item A supermassive black hole is thought to lie at the center of these ancient galaxies. These gluttonous giants consume gas and dust, and may play a role in the slower growth of elliptical galaxies.
	\end{itemize}
	\\
	\hline
\end{longtabu}

\subsubsection{Dwarf Elliptical Galaxy}
\begin{longtabu} to \textwidth{| | p{3.5 cm} | X | | }
\hline
General Characteristics &
 elliptical galaxies that are smaller than ordinary elliptical galaxies. They are quite common in galaxy groups and clusters, and are usually companions to other galaxies.