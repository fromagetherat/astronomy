\section{Globular Clusters}
\subsection{Composition}
\begin{itemize}[noitemsep]
	\item a spherical collection of stars orbiting a galactic core as a satellite
	\item the force of gravity maintains the spherical shape and gives them a high stellar density towards their center
	\item hundreds of thousands of old, low-metal stars --- there is no gas or dust, implying it was all turned into stars
		\begin{itemize}[noitemsep]
			\item Globular clusters can contain a high density of stars; on average about 0.4 stars per cubic parsec, increasing to 100 or 1000 stars per cubic parsec in the core of the cluster. The typical distance between stars in a globular cluster is about 1 light year.
			\item normally Population II stars w/ low proportion of elements other than hydrogen or helium (i.e. low metallicity)
			\item \textbf{Oosterhoff groups} --- 2 populations of globular clusters
				\begin{itemize}[noitemsep]
					\item group 1 --- ``metal-rich," has a slightly stronger metallic spectral line
					\item group 2 --- ``metal-poor," has a slightly weaker metallic spectral line w/ a slightly longer period of RR Lyrae variable stars
					\item  Many scenarios have been suggested to explain these subpopulations, including violent gas-rich galaxy mergers, the accretion of dwarf galaxies, and multiple phases of star formation in a single galaxy. In the Milky Way, the metal-poor clusters are associated with the halo and the metal-rich clusters with the bulge.
				\end{itemize}
		\end{itemize}
	\item the high star density leads to close interactions and near-collisions of stars often $\rightarrow$ exotic classes of stars are very common in globular clusters
		\begin{itemize}[noitemsep]
			\item e.g.  blue stragglers, millisecond pulsars and low-mass X-ray binaries
			\item searches for black holes in the center of globular clusters have demonstrated evidence for a new kind of black hole intermediate b/w the standard black hole and the supermassive. the mass of these intermediate black holes is proportionate the mass of the clusters.
			\item however, this is disputed b/c the heaviest objects in globular clusters will migrate to the center due to the effects \gls{mass segregation}*. therefore, the sharp increase in mass-to-light ratio toward the center of a cluster may be possible without the presence of a black hole.
		\end{itemize}
\end{itemize}
\subsection{Formation}
\begin{itemize}[noitemsep]
	\item  it remains uncertain whether the stars in a globular cluster form in a single generation or are spawned across multiple generations over a period of several hundred million years
	\item  In many globular clusters, most of the stars are at approximately the same stage in stellar evolution, suggesting that they formed at about the same time. However, the star formation history varies from cluster to cluster, with some clusters showing distinct populations of stars.
	\item it is also theorized that globular clusters w/ variation in their star populations formed from the merging of multiple clusters, which is consistent w/ Hubble Telescope Observations of massive clusters of clusters in close proximity
	\item clusters typically arise in regions of efficient star formation w/ interstellar medium of a higher density than in normal star-forming regions
	\item Research indicates a correlation between the mass of a central supermassive black holes (SMBH) and the extent of the globular cluster systems of elliptical and lenticular galaxies. The mass of the SMBH in such a galaxy is often close to the combined mass of the galaxy's globular clusters.
	\item No known globular clusters display active star formation, which is consistent with the view that globular clusters are typically the oldest objects in the Galaxy, and were among the first collections of stars to form. Very large regions of star formation known as super star clusters, such as Westerlund 1 in the Milky Way, may be the precursors of globular clusters.
\end{itemize}