\subsubsection{Lenticular Galaxies}
\begin{longtabu} to \textwidth{| | p{4cm} | X | |}
\hline
	General Characteristics &
	a type of galaxy intermediate between an elliptical (denoted E) and a spiral galaxy in galaxy morphological classification schemes, they are also known as $S_0$ galaxies. they contain large discs but do not have large spiral arms; they are disc galaxies that have used up or lost most of their interstellar matter and therefore have very little ongoing star formation. however, they retain significant dust in their disks. As a result, they consist mainly of aging stars (like elliptical galaxies). 
	\\
	\hline
	Shapes and Sizes &
	\begin{itemize}[noitemsep]
		\item visible disk component w/ large bulge 
		\item much higher bulge-to-disk ratios than typical spirals
		\item may exhibit central bar
		\item lenticular galaxies can be classified according to the $S0_n$ and $SB0_n$ systems
			\begin{itemize}[noitemsep]
				\item in the $S0_n$ system, $n$ indicates the amount of dust present
				\item in the $SB0_n$ system, $n$ indicates the prominence of the bar. $SB0_1$ galaxies have the least defined bar structure and are only classified as having slightly enhanced surface brightness along opposite sides of the central bulge. The prominence of the bar increases with index number, thus $SB0_3$ galaxies have very well defined bars that can extend through the transition region between the bulge and disk.
			\end{itemize}
	\end{itemize}
	\\
	\hline
	Celestial Bodies &
	In many respects the composition of lenticular galaxies is like that of ellipticals. For example, they both consist of predominately older, hence redder, stars. All of their stars are thought to be older than about a billion years, in agreement with their offset from the Tully?Fisher relation. In addition to these general stellar attributes, globular clusters* are found more frequently in lenticular galaxies than in spiral galaxies of similar mass and luminosity. They also have little to no molecular gas (hence the lack of star formation) and no significant hydrogen $\alpha$ or 21-cm emission. Finally, unlike ellipticals, they may still possess significant dust.
	\\
	\hline
	Evolution & 
	\begin{enumerate}
		\item The absence of gas, presence of dust, lack of recent star formation, and rotational support are all attributes one might expect of a spiral galaxy which had used up all of its gas in the formation of stars
			\begin{enumerate}
				\item anemic* spiral galaxies are similar to lenticular galaxies if the spiral pattern were dispersed
				\item according to the Tully-Fisher relation, spiral galaxies and lenticular galaxies have the same slope on the luminosity/absolute magnitude axis, but are offset by $\Delta I = 1.5$. This implies that lenticular galaxies were once spiral galaxies but are now dominated by old, red stars.
			\end{enumerate}
		\item it has also been postulated that lenticular galaxies form from galaxy mergers.
			\begin{enumerate}
				\item  lenticular galaxies typically have surface brightness much greater than other spiral classes. It is also thought that lenticular galaxies exhibit a larger bulge-to-disk ratio than spiral galaxies and this may be inconsistent with simple fading from a spiral. they also have an increased frequency of globular clusters*.
				\item Mergers are unable to account for the offset from the Tully?Fisher relation without assuming that the merged galaxies were quite different from those we see today.
				\item advanced models of the central bulge indicate that it is smaller, lessening the inconsistency.
			\end{enumerate}
	\end{enumerate}
	\\
	\hline
\end{longtabu}
	
	