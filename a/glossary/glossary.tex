\newglossaryentry{isophote}
{
	name={isophote},
	description={a curve on an illuminated surface that connects points of equal brightness. In astronomy, the isophote is commonly used to define two things: the shape of an object, and the amount of light it gives off. The most common use of isophotes in astronomy is in the imaging and classification of galaxies, particularly of elliptical galaxies. The isophotes of elliptical galaxies provide information on a galaxy's shape, and hence upon its structure and dynamical behavior. Isophotes can be used on spiral galaxies, too, particularly to measure their radii, or to map the structures within their spiral arms. Isophotes are also used to measure the size, structure, and brightness of many gaseous or tenuous objects, such as X-ray galaxy clusters, radio jets from quasars, and the distribution of dust in our Galaxy. They have even been used to map the light reflected from the Moon and other planets to understand the properties of their surfaces}
}