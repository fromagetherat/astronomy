\newglossaryentry{isophote}
{
	name={isophote},
	description={a curve on an illuminated surface that connects points of equal brightness. In astronomy, the isophote is commonly used to define two things: the shape of an object, and the amount of light it gives off. The most common use of isophotes in astronomy is in the imaging and classification of galaxies, particularly of elliptical galaxies. The isophotes of elliptical galaxies provide information on a galaxy's shape, and hence upon its structure and dynamical behavior. Isophotes can be used on spiral galaxies, too, particularly to measure their radii, or to map the structures within their spiral arms. Isophotes are also used to measure the size, structure, and brightness of many gaseous or tenuous objects, such as X-ray galaxy clusters, radio jets from quasars, and the distribution of dust in our Galaxy. They have even been used to map the light reflected from the Moon and other planets to understand the properties of their surfaces}
}
\newglossaryentry{absolute magnitude}
{
	name={absolute magnitude},
	description={a measure of the luminosity of a celestial object, on a logarithmic astronomical magnitude scale. An object's absolute magnitude is defined to be equal to the apparent magnitude that the object would have if it were viewed from a distance of exactly 10 parsecs (32.6 light-years), with no extinction (or dimming) of its light due to absorption by interstellar dust particles. By hypothetically placing all objects at a standard reference distance from the observer, their luminosities can be directly compared on a magnitude scale}
} 
\newglossaryentry{galaxy harassment}
{
	name={galaxy harassment},
	description={a type of interaction between a low-luminosity galaxy and a brighter one that takes place within rich galaxy clusters, such as Virgo and Coma, where galaxies are moving at high relative speeds and suffering frequent encounters with other systems of the cluster by the high galactic density of the latter. According to computer simulations, the interactions convert the affected galaxy disks into disturbed barred spiral galaxies and produces starbursts followed by, if more encounters occur, loss of angular momentum and heating of their gas. The result would be the conversion of (late type) low-luminosity spiral galaxies into dwarf spheroidals and dwarf ellipticals}
}
\newglossaryentry{globular cluster}
{
	name={globular cluster},
	description={a spherical collection of stars that orbits a galactic core as a satellite. Globular clusters are very tightly bound by gravity, which gives them their spherical shapes and relatively high stellar densities toward their centers. they are found in the halo of the galaxy. Every galaxy of sufficient mass in the Local Group has an associated group of globular clusters, and almost every large galaxy surveyed has been found to possess a system of globular clusters. Although it appears that globular clusters contain some of the first stars to be produced in the galaxy, their origins and their role in galactic evolution are still unclear. It does appear clear that globular clusters are significantly different from dwarf elliptical galaxies and were formed as part of the star formation of the parent galaxy rather than as a separate galaxy}
}
\newglossaryentry{open cluster}
{
	name={open cluster},
	description={a group of up to a few thousand stars that were formed from the same giant molecular cloud and have roughly the same age, found in the disk of a galaxyThey are loosely bound by mutual gravitational attraction and become disrupted by close encounters with other clusters and clouds of gas as they orbit the galactic center. This can result in a migration to the main body of the galaxy and a loss of cluster members through internal close encounters. Open clusters generally survive for a few hundred million years, with the most massive ones surviving for a few billion years. In contrast, the more massive globular clusters of stars exert a stronger gravitational attraction on their members, and can survive for longer. Open clusters have been found only in spiral and irregular galaxies, in which active star formation is occurring}
}
\newglossaryentry{galactic plane}
{
	name={galactic plane},
	description={the plane on which the majority of a disk-shaped galaxy's mass lies. The directions perpendicular to the galactic plane point to the galactic poles}
}
\newglossaryentry{mass segregation}
{
	name={mass segregation},
	description={the process by which heavier members of a gravitationally bound system, such as a star cluster or cluster of galaxies, tend to move toward the center, while lighter members tend to move farther away from the center}
}